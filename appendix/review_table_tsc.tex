\section{Time-Series Clustering}

\begin{longtable}{p{0.03\textwidth}p{0.17\textwidth}p{0.25\textwidth}p{0.25\textwidth}p{0.2\textwidth}}
    \hline
    Ref. & Representation & Similarity measure & Clustering Algorithms & Evaluation \\
    \hline 
    \hline
    \cite{mixture_gaussian_hmm} 		    & Mixture Gaussian hidden Markov model (MGHMM).& & Expectation-maximization (EM).& Bayesian information criterion (BIC).\\ \hline
    \cite{financial_tsc_variance_ratio}	    & Variance ratio statistics.& Euclidean distance.& Hierarchical clustering mainly, and K-means.& Duda-Hart $Je(2)/Je(1)$ indices.\\ \hline
    \cite{hmm_pm10_quantifying_impacts}     & HMM. States correspond to concentration regimes.& Which state each HMM is in.& Cluster together TS with corresponding HMMs in the same state.& \\ \hline
    \cite{tsc_rev}                          & This is a review of time series clustering.& & & \\ \hline
    \cite{ghsom_optimal_hedge_ratio}        & Raw TS and some extracted statistics: variance, covariance, spread and differences.& & Growing hierarchical self-organizing map (GHSOM).& \\ \hline
    \cite{hier_clust_w_state_space_models}  & Linear combinations of spline basis functions.& Euclidean distance.& State space modelling, K-means and complete-linkage hierarchical clustering.& L-curve and gap statistic.\\ \hline
    \cite{topology_for_shape_based_tsc}     & EMD for filtering out stochastic components, then extract topological features.& Euclidean distance.& K-means.& Precision, recall, F1-score and Matthews correlation coefficient.\\ \hline
    \cite{community_detection_networks_tsc} & Construct network between time series using dissimilarity matrix. Use KNN, and $\epsilon$-NN to create networks from matrix.& Test a multitude of different distance functions. DTW performs best.& Test many community detection algorithms to sort network into clusters.& Rand index.\\ \hline
    \cite{clust_large_datasets_aghabozorg}  & Symbolic Aggregate ApproXimation (SAX).& Approximate distance (APXDIST), Euclidean distance, and DTW.& Custom three step algorithm (MTC), with preclustering, sub-clustering, and merging to form final clusters.& Accuracy, precision, recall and F-measure.\\ \hline
    \cite{garch_robust_tsc}                 & Generalized autoregressive conditional heteroskedasticity (GARCH) model.& Tests different metrics based on squared Euclidean distance between unconditional volitility and time varying volitility.& Tests different variations of fuzzy C-medoids. & Xie-Beni index, and Fuzzy Rand index.\\ \hline
    \cite{BSLEX_nonlin_nonstat_tsc}         & Bispectral Smoothed Localized Complex EXponential (BSLEX) algorithm.& Aggragated quasi-distance between smoothed bispectra across blocks.& Agglomorative hierarchical clustering with Ward's linkage.& Silhuette index as stopping criterion, and Rand Index, entropy and purity to evaluate cluster effectiveness.\\ \hline
    \cite{multivariate_tsc_hmm}             & HMMs.& Kulback-Leibler distance between the likelihood of observation sequency $T$ given HMM $\lambda$.& K-medoids.& Silhouette index, Bavies-Bouldin index and Dunn index.\\ \hline
    \cite{copula_fuzzy_tsc_spatial}         & Copula-based model for time series.& P-norm of difference between copula of two points, and upper bound copula.& Fuzzy C-medoids.& Fuzzy Silhouette (FS) index, adjusted Rand index (ARI), fuzzy Rand index (FRI).\\ \hline
    \cite{shape_feat_mod_tsc_rfa}           & DWT, SAX and AR model.& MINDIST, Euclidean, Minkowski, Pearson correlation coefficient and DTW distance. & Agglomerative Hierarchical clustering with Ward linkage.& Uses the clusters produced to perform regional frequency analysis, and then evaluates model using bias, RMSE, relative RMSE (RRMSE) and Nash criterion.\\ \hline
    \cite{ica_tsc_sea_level}                & Independent component analysis (ICA).& Not specified.& Hierarchical clustering with complete linkage.& \\ \hline
    \cite{apxdist_sax_k_modes}              & SAX.& APXDIST between symbolic representations of TS.& Extended version of K-modes.& SSE for stopping criteria, Rand index, Normalized Mutual Information (NMI), Purity, Jaccard, F-measure, Folks and Mallows (FM) and entropy.\\ \hline
    \cite{multivar_tsc_riemann_manifold}    & Tranforms the covariance matrices of the time series into a tangent space.& Euclidean distance.& Hierarchical clustering with average linkage.& \\ \hline
    \cite{tsc_total_variation_distance}     & Normalized spectral densities.& Total variation distence.& Agglomorative hierarchical clustering with complete and average linkage.& Dunn's index.\\ \hline
    \cite{temporal_tsc_threshold_ar_models} & Self-exciting threshold autoregressive model (SETAR).& Primarily tests Euclidean distance, Haussdorf distance and DTW, but, tests 22 different ones.& Primary method is spectral clustering, but also tests K-medoids, and fuzzy C-means.& Measures accuracy of method on clustering simulated data, and uses Gap statistic as stopping criterion.\\ \hline
    \cite{tsc_ar_metric_air_pollution}      & AR model.& A type of exponential Euclidean distance.& Fuzzy C-medoids.& Fuzzy Silhouette index.\\ \hline
    \cite{wavelet_multivar_tsc_multi_pca}   & Continous wavelet transform (CWT).& Multi-scale PCA similarity matric.& Fuzzy C-means.& Precision and recall of classification according to labels, and silhouette index.\\ \hline
    \cite{stock_price_tsc_regr_trees_som}   & Preprocessing using Hodrick-Prescott (HP) filter, feature extraction using state space models or regression trees.& & Self-organizing map (SOM).& Silhouette index as stopping criterion.\\ \hline
    \cite{ar_metric_trimmed_fuzzy_tsc_pm10} & ARIMA model.& Euclidean distance between AR weights.& Trimmed fuzzy C-medoids.& Decides number of clusters by looking at the rate of decrease, and second derivative of an objective function with regard to a trimming factor $\alpha$.\\ \hline
    \cite{dependency_tsc_energy_markets}    & Permutation based coding of time series.& Use four distance metrics based on mutual information, entropy and Cramer's V association measure.& Hierarchical clustering with single, complete and average linkage.& \\ \hline
    \cite{moar_mpl_tsc}                     & Mixture of autoregressions (MoAR) models.& & Maximum pseudolikelihood esitmation using EM algorithm. & \\ \hline
    \cite{copula_ica_tsc}                   & Use PCA and custom ICA algorithm for feature extraction.& Euclidean distance between extracted features.& Hierarchical clustering with average, single, complete and Ward linkage, and K-means.& CH, Friedman, C-index, Dunn's, SDbw and Silhouette index.\\ \hline
    \cite{tsc_slaughterhouse}               & Extracts various signal statistics, and performs feature extraction using R package ''Psych''.& Euclidean distance.& Hierarchical clustering with complete linkage.& \\ \hline
    \cite{ambient_air_vape_k_means}         & DWT with the Haar wavelet, and a globel sensitivity analysis.& Euclidean distance, to minimize variance.& K-means.& \\ \hline
    \cite{multivar_tsc_community_detection} & Multi-relational network in topological domain, static (time-invariant), and dynamic (time-varying).& & Multi-nonnegative matrix factorization. Compares their approach to three other community detection algorithms.& Rand index, adjusted Rand index and purity.\\ \hline
    \cite{fuzzy_c_means_pso_svd}            & Uses Singular value decomposition (SVD) to represent the cluster centroids.& Pearson correlation coefficient.& Fuzzy C-means with particle swarm optimization.& Precision, and F-measure.\\ \hline
    \cite{hysteresis_tsc_tensor_decomp}     & Multivariate time series are transformed into 3-order hysteresis tensors, then multilinear PCA is used to reduce dimensionality.& Tensor distance metric. Cluster centers initialized based on cycle feature variation.& Tensor K-means (CTKmeans).& RI, ARI, Jaccard coefficient and FM index.\\ \hline
    \cite{comp_many_model_based_tsc_GMM}    & Compares ten model-based clustering methods. GMM and Markov-switching model perform best.& & Expectation-maximization (EM). & Misclassification rates.\\ \hline
    \cite{dwt_hac_kmeans_som}               & Vari-segmented DWT.& Euclidean distance.& K-means, hierarchical agglomerative clustering and SOM.& \\ \hline
    \cite{xml_dft_delaunay_traingulation}   & Discrete Fourier transform (DFT).& Euclidean distance.& Delaunay Triangulation method.& Purity and F-measure.\\ \hline
    \cite{svd_birch_tsc_stock_price}        & Piecewise SVD, and piecewise aggregate approximation (PAA).& Euclidean Distance.& Balanced Iterative Reducing and Clustering using Hierarchies (BIRCH).& \\ \hline
    \cite{road_grade_china_pca_kmeans}      & Extracts signal statistics, and uses PCA for feature selection.& Euclidean distance.& K-means.& Analyses the correlations of specific features with different clusters.\\ \hline
    \cite{fragmented_periodogram}           & Fragmented periodogram.& Euclidean distance.& Spectral clustering.& \\ \hline
    \cite{auto_encoder_many_tsc_algorithms} & Extract statistical features of time series, then use a convolutional auto-encoder for further feature extraction.& Mainly Euclidean distance for the model-based approach.& K-means, hierarchical clustering with Ward linkage, spectral clustering, Gaussian mixture models, BIRCH, affinity propagation, mean shift, DBSCAN. &Adjusted Rand index. \\ \hline
    \cite{load_tsc_state_space_model}       & Compares a feature-based approach using PCA, with a model-based approach using state-space models for the individual time series.& Inverse exponential Euclidean distance for feature based approach, and Euclidean distance for model-based approach.& & Silhuette index as stopping criterion.\\ \hline
    \cite{struct_damage_ar_fuzzy_c_means}   & AR model.& Euclidean distance.& Fuzzy C-means.& \\ \hline
    \cite{fstar_hac_tsc}                    & Flexible space-time autoregressive (FSTAR) models.& Use Wald statistic to compare model parameters of univariate STAR models, and p-value as a similarity metric.& Hierarchical agglomorative clustering.& ARI.\\ \hline
    \cite{multivariate_tsc_common_pca}      & Common PCA (CPCA).& Cluster centroids represented by common projection axis of all TS in a specific cluster, then reconstruction error of time series using cluster centroid used as similarity metric.& Custom algorithm, similar to K-means.& Precision.\\ \hline
    \cite{tensor_multi_elastic_kernel_tsc}  & Map the time series to multiple high-dimensional tensors using multiple kernals.& Matrix $L^p$-norms.& Self-developed multi kernal clustering algorithm (MKC).& NMI and Rand Index (RI).\\ \hline
    \cite{var_multivar_tsc}                 & Vector autoregressive (VAR) models.& Euclidean distance.& Test two self-developed algorithms based on performing statistical test of whether TS come from same DGN. & Purity index.\\ \hline
    \hline
    \caption{Summary of model based time-series clustering methods}
    \label{tab:machine_learning_wt_cm_summary}
\end{longtable}

