\chapter{Appendices}

\section{Machine learning for wind turbine condition monitoring}

\begin{longtable}{p{0.03\textwidth}p{0.2\textwidth}p{0.3\textwidth}p{0.4\textwidth}}
    \hline
    Ref. & Input & Feature extraction method & Machine learning model \\
    \hline \hline
    \cite{ml_cm_wt_blade_ARMA_2018} & Vibration signal. & ARMA model and J48 decision tree & Tests a set of (38) meta-, misc-, rule- and tree-based classifiers for fault detection in blades. \\  \hline
    \cite{AD_and_fault_analysis_wt_DAE} & SCADA data. & & Deep autoencoder made of Restricted Boltzman Machines (RBMs) to model normal behaviour of SCADA variables (gearbox and generator temperature). Uses $E_e$ for anomaly detection, with adaptive threshold set using extreme function theory (EFT). \\ \hline
    \cite{unsup_graphical_modeling_wt_cm} & SCADA data. & Spatiotemporal pattern network & Unsupervised use of RBMs for anomaly detection. \\ \hline
    \cite{fault_detection_and_isolation_using_classifier_fusion} & SCADA data. & Time-frequency domain analysis, DWT and ARMA model & Uses fusion of several classifiers for fault detection in a wind turbine.  \\ \hline
    \cite{lin_and_non_lin_feat_for_ice_detection_on_blades} & Ultrasonic testing. & Tests linear and non-linear PCA and ARMA models & Neighbourhood component analysis for feature selection. Tests 20 different supervised classifiers for detecting ice on blades. \\ \hline
    \cite{perf_mon_of_wt_using_extreme_func_theory} & Wind-Power curve. &  & Uses Gaussian Process regression with EFT to determine whether a particular power curve is an outlier.\\ \hline
    \cite{blade_damage_detection_sup_ml_alg} & Acustic emission. & FFT. & Uses Distinguishability Measure for feature selection, and logistic regression and SVC for binary blade fault classification. \\ \hline
    \cite{wt_cm_using_cloud_computing_and_HELM} & Power signal, wind speed and ambient temperature. & & Hierarchical Extreme Learning Machine (H-ELM) for detection of anomolous behaviour. \\ \hline
    \cite{GP_operational_curve_monitoring} & SCADA data. & & Gaussian processes regression to estimate wind-power curve \\ \hline
    \cite{high_freq_scada_perf_monit_sensitivity} & SCADA data. & & Tests KNN, random forest, and SVR to estimate power curve. Detects anomalies by $E_e$. \\ \hline
    \cite{dict_learning_monitor_wt_drivetrain_bearing} & Vibration signals. & & Uses unsupervised dictionary learning extracting features which are then used to determine fault in drivetrain bearings. \\ \hline
    \cite{ANN_damage_detection_gearbox_wt} & Oil temperature, wind speed, rotor speed and active power. & & Trains ANN to estimate vibration signal, uses $E_e$ for anomaly detection. \\ \hline
    \cite{multiway_PCA_multivar_inference_cm_wt} & SCADA data. & PCA. & Sets up baseline model using multiway PCA, then finds outliers by hypothesis testing whether multivariate distribution is equal to baseline. \\ \hline
    \cite{image_texture_analysis_FD_wt} & FAST wind turbine simulator. & Image texture analysis tools. & KNN, Linear Discriminant Model, decision trees, bag-tree, linear SVC.\\ \hline
    \cite{auto_associative_nn_wt_fault_detection} & SCADA data. & K-means for outlier elimination. & Uses Auto-Associative Neural Networks as an autoencoder, and the Hotel T2 statistic as a dynamic threshold for the $R_e$.\\ \hline
    \cite{abnormal_detection_scada_data_mining} & SCADA data. & Grey correlation algorithm for eigenvector extraction. & Use genetic algorithm for feature selection, and SVR for estimating performance curves (active power, rotor speed and blade pitch angle). \\ \hline
    \cite{health_cond_model_nn_proportional_hazard_models} & SCADA data. & & Uses three NN for normal behaviour modelling of rotor speed, gearbox temperature and generator temperature. $E_e$ sendt to proportional hazard model which sets dynamic threshold. \\ \hline
    \cite{ice_detection_using_ITL} & SCADA data. & & Uses Inductive Transfer Learning and five differen ML classifiers for ice detection on blades. \\ \hline
    \cite{VMD_MPE_COVAL_fault_detection_gearbox} & Vibration signals. & Variational mode decomposition (VMD). & Uses multi-scale permutation entropy (MPE) used for feature selection, COVAL for domain normalization and an SVC for binary fault classification.\\ \hline
    \cite{SVR_blade_pitch_curve_cm} & SCADA data. & & Uses bins and SVR to estimate blade angle pitch curve. \\ \hline
    \cite{detecting_malfunctions_wt_generator_bearings_generic_vs_specific_models} & SCADA data. &  & Tests different architectures of ANNs to for estimating temperature of non-drive end bearing. Uses $E_e$ for anomaly detection. \\ \hline
    \cite{image_based_surface_damage_detection_DL_drone_inspection} & Images taken by drones & & Recurrent Convolutional Neural Network to classify structural damage in blades.\\ \hline
    \cite{image_based_YOLO_YSODA} & Uses images taken from ground level & & Convolutional Neural Network, and YOLO-based small object detection approach (YSODA) for damage detection in blades and hub.\\ \hline
    \cite{vibration_acustic_decision_tree_SVM_gearbox} & Vibration signals, acustic emission and oil particle analysis & DWT. & Uses decision tree for feature selection, and SVC for assessing fault severity in gearbox\\ \hline
    \cite{AI_image_analytics_2_classify_blade_defects} & Uses images taken from ground level & & Convolutional Neural Network to detect cracks $\&$ damage in blades.\\ \hline
    \cite{dirt_n_mud_detection_using_guided_waves} & Ultrasonic testing & PCA and ARMA models. & Neighbourhood Component analysis for feature selection and an ensamble of KNN, linear SVC, decision trees, LDA and subspace discriminant to estimate amount of dirt and mud on blades. \\ \hline
    \cite{fault_classification_using_CSO_SVM} & Uses pitch position, rotor speed and generator speed. & & Detects faults with an SVC with parameters optimized by Cuckoo-swarm optimization. \\ \hline
    \cite{vibration_ARMA_decision_tree_cm_wt} & Vibration signals & ARMA model. & Dominating features selected with J48 decision tree, fault classification done with Bayesian- and lazy classifiers. \\ \hline
    \cite{integrated_cm_bearing_fault_wt_gearbox} & Vibration signals, acustic emission and oil particle analysis & DWT and PCA. & Dominating features selected with decision tree, fault detection done with SVC. \\ \hline
    \cite{blade_defect_detection_imaging_array} & Images taken from imaging array & & Uses a deep neural network for binary classification of blade defects. \\ \hline
    \cite{unsupervised_AD_blade_damage_deep_features_images} & Uses images taken by drone & Uses a CNN trained on an unrelated image dataset to extract general features. & Compress features with PCA, and pass them to a unsupervised one-class SVM. \\ \hline
    \cite{RF_XGB_fault_detection} & Uses the FAST wind turbine simulator to get SCADA data. & Random forest. & Uses XGBoost to train an ensamble of classifiers for specific faults. \\ \hline
    \cite{CBPM_ABPM_maintainance_model} & SCADA data. & & Uses several ANN to build a normal behaviour model of temperature in gearbox and high speed shaft, then uses $E_e$ together with the age of the age of the turbine to predict anomolous behaviour.\\ \hline
    \cite{improved_power_curve_monitoring_of_wt} & SCADA data. & & Uses Pearson product-moment rank correlation to select features, and applies different ANN structures to predict the active power. \\ \hline
    \cite{DBN_simulink_SCADA_FD} & SCADA data from a simulink model of a wind turbine. & & Uses DBN for detecting anomolous behaviour. First traines individual RBMs to recreate input, and then uses labeled data to fine tune DBN to detect faults. \\ \hline
    \cite{wt_gearbox_bearing_temp_KS_CNN} & SCADA data. &  & Uses kolmogorov-smirnov test to compare different turbines at same moment in time combined with the $E_e$ of the gearbox bearing temperature of an ANN to detect anomalies.\\ \hline
    \cite{fault_monitoring_HMM} & Vibration signals. & Approximates vibration distributions at different rotor speeds with Weibull distribution. & Uses a HMM for statistical fault detection. \\ \hline
    \cite{model_based_fuzzy_logic_cm_wt} & SCADA data. & & Compares linear models, ANNs and state-dependent parameter models for fault detection. \\ \hline
    \cite{fault_detect_PARAFAC_k_means} & SCADA data. & parallel factor analysis (PARAFAC) as a decomposition method & uses K-means clustering after decomposition for fault detection. \\ \hline
    \cite{online_fd_using_PCA_different_operating_zones} & Uses a wind turbine simulator for SCADA data. & & Multiple PCA models are as a statistical reference reflecting the data variability in local zones and used in parallel for online fault detection. \\ \hline
    \cite{roller_bearings_cm_fisher_score_and_permutation_entropy} & Vibration signals. & Variational mode decomposition & Uses Fisher score and ReliefF algorithm for feature selection. Feeds selected signals into a multi-class SVC for bearing fault detection. \\ \hline
    \cite{deep_learning_for_imbalanced_class_detection_bearing_cm} & SCADA data. & & Uses deep neural networks for detection of icing on the blades. \\ \hline
    \cite{reliability_analysis_detecting_FA_NN_wt} & SCADA data. & & Combines NN with alarms generated by SCADA system to reduce false alarm rate. \\ \hline
    \cite{DBN_chicken_swarm_optim} & SCADA data. & & Uses K-means clustering to partition turbines into different operating states, and a specific DBN of RBMs for each cluster to forecast the gearbox main bearing temperature. Uses $E_e$ to detect anomalies, threshold set by Mahalanobis distance. \\ \hline
    \cite{vibration_fault_diagnosis_wt_planetary_gearbox} & & & This is a literary review of vibration based condition monitoring and fault diagnosis of planetary gearboxes in wind turbines. \\ \hline
    \cite{ml_for_wt_cond_monit_rev} & & & This is a literary review of machine learning methods used for condition monitoring of wind turbines. \\ \hline
    \hline
    \caption{Summary of machine learning methods for wind turbine condition monitoring}
    \label{tab:machine_learning_wt_cm_summary}
\end{longtable}