\section{Code}
To produce all the relevant plots all the listings can be posted and run in the same file, to produce all the relevant plots used in this report.
They are split up into separate listings such that it is easier to link each listing to each plot.

\begin{listing}[h]
\begin{minted}
[
frame=lines,
framesep=2mm,
baselinestretch=1.2,
fontsize=\footnotesize,
linenos
]
{Python}
import pandas as pd
from sklearn.decomposition import PCA
from datetime import datetime
import time
import timeit
import numpy as np
import matplotlib.pyplot as plt
import copy

work_directory = 'Not relevant'
data_sets_path = \
'data_sets/big_turbine_data/'
filename = 'big_turbine_data'
path_pictures = work_directory + 'pictures/'

completeDF = pd.read_csv(work_directory+data_sets_path+filename+'.csv', 
                         delimiter=',', 
                         skiprows=1)

completeDF['Time'] = pd.to_datetime(completeDF['Time'], format='%Y-%m-%dT%H:%M:%S.%fZ')
completeDF.set_index('Time', inplace=True)
\end{minted}
\label{lst:load_modules_and_data}
\caption{Module fetching and loading data}
\end{listing}

\begin{listing}
\begin{minted}
[
frame=lines,
framesep=2mm,
baselinestretch=1.2,
fontsize=\footnotesize,
linenos
]
{Python}
# Dropping max/min cols
column_names = completeDF.columns
for name in column_names:
    if ('Max' in name) or ('Min' in name):
        completeDF.drop(name, axis=1, inplace=True)
        
# Renaming columns
column_names = completeDF.columns
new_col_names = []
for name in column_names:
    split_name = name.split('/')
    turbine_num = split_name[2]
    if split_name[-1] == 's)':
        value_meas = split_name[-3]
    else:
        value_meas = split_name[-1]
    new_name = turbine_num + '/' + value_meas
    split_new = new_name.split('(Avg)')
    new_col_names += [split_new[0] + split_new[1]]   
completeDF.columns = new_col_names

# Create a dictionary of windturbine dataframes
column_names = completeDF.columns
dict_of_turbine_dfs = {}
for i in range(1,14):
    turbine_num = 'WTUR' + str(i) + '/'
    tempDF = pd.DataFrame()
    for name in column_names:
        if turbine_num in name:
            tempDF[name] = completeDF[name]    
	dict_of_turbine_dfs[turbine_num[:-1]] = tempDF

# Normalize values from each dataframe
dict_norm_dfs = {}
for turb_name in list(dict_of_turbine_dfs.keys()):
	df = dict_of_turbine_dfs[turb_name]
	dict_norm_dfs[turb_name] = (df - np.min(df.values, axis=0)) / (np.max(df.values, axis=0) - np.min(df.values, axis=0))
\end{minted}
\label{lst:form_norm}
\caption{Data formatting and normalization}
\end{listing}

\begin{listing}
\begin{minted}
[
frame=lines,
framesep=2mm,
baselinestretch=1.2,
fontsize=\footnotesize,
linenos
]
{Python}
dict_of_turbine_dfs['WTUR4'].plot(figsize=(12,8),subplots=True, layout=(2,2))
plt.legend(fontsize=10)
name = path_pictures + 'one_turbine_all_vals.png'
plt.savefig(fname=name)
\end{minted}
\label{lst:data_one_wt}
\caption{Code used to produce plot in figure \ref{fig:data_one_wt}}
\end{listing}

\begin{listing}
\begin{minted}
[
frame=lines,
framesep=2mm,
baselinestretch=1.2,
fontsize=\footnotesize,
linenos
]
{Python}
# PCA Finding portions of explained variance
cumulative_explained_variance = {}
for turb in list(dict_norm_dfs.keys()):
    X = dict_norm_dfs[turb].values
    pca = PCA().fit(X)  # project from 64 to 2 dimensions
    cumulative_explained_variance[turb] = np.cumsum(pca.explained_variance_ratio_)

# Figure XX Plot explained variance for a given number of principal components
plt.figure(figsize=(10,8))
for turb in list(cumulative_explained_variance.keys()):
    plt.plot(cumulative_explained_variance[turb], label=turb)
    
# plt.title('Explained variance vs. Number of PCA components', fontsize=18)
plt.xlabel('Number of components', fontsize=14)
plt.ylabel('Cumulative explained variance',fontsize=14);
plt.legend(fontsize=10)

name = path_pictures + 'cumulative_explained_variance.png'
plt.savefig(fname=name)
\end{minted}
\label{lst:cum_exp_var}
\caption{Code used to calculate cumulative explained variance, and produce plot in figure \ref{fig:cum_exp_var}}
\end{listing}

\begin{listing}
\begin{minted}
[
frame=lines,
framesep=2mm,
baselinestretch=1.2,
fontsize=\footnotesize,
linenos
]
{Python}
mse = {}
for turb in list(dict_norm_dfs.keys()):
    Xstd = dict_norm_dfs[turb].values
    pca = PCA(n_components=2)
    pca.fit(Xstd)
    P=pca.components_
    T = Xstd.dot(P.T)
    Xhat = np.dot(T[:,0].reshape(-1,1),P[0,:].reshape(-1,1).T)
    mse[turb] = (np.square(Xstd - Xhat)).sum(axis=1)

# Figure XX Plot of reconstruction error using principal components
plt.figure(figsize=(12,8))
for turb in list(mse.keys()):
    plt.plot(mse[turb], label=turb)
    
# plt.title('Reconstruction Error when using two principal components', fontsize=18)
plt.xlabel('Time', fontsize=14)
plt.ylabel('Mean square reconstruction error',fontsize=14);
plt.legend(fontsize=10)
name = path_pictures + 'reconstruction_error.png'
plt.savefig(fname=name)
\end{minted}
\label{lst:recon_error}
\caption{Code used to calculate reconstruction error, and produce plot in figure \ref{fig:recon_error}}
\end{listing}

\begin{listing}
\begin{minted}
[
frame=lines,
framesep=2mm,
baselinestretch=1.2,
fontsize=\footnotesize,
linenos
]
{Python}
# Figure XX Plot of values from perturbed WT versus non-perturbed
dict_perturbed_norm_dfs = copy.deepcopy(dict_norm_dfs)
dict_perturbed_norm_dfs['WTUR12'].columns
for idx in range(9000,11000):
    dict_perturbed_norm_dfs['WTUR12'].iat[idx,0] = 1
    dict_perturbed_norm_dfs['WTUR12'].iat[idx,2] = 0
    
df_to_plot = pd.DataFrame()    
df_to_plot['Power perturbed'] = dict_perturbed_norm_dfs['WTUR12']['WTUR12/ActivePower (kW)']
df_to_plot['Speed perturbed'] = dict_perturbed_norm_dfs['WTUR12']['WTUR12/WindSpeed1']
df_to_plot['Power'] = dict_norm_dfs['WTUR12']['WTUR12/ActivePower (kW)']
df_to_plot['Speed'] = dict_norm_dfs['WTUR12']['WTUR12/WindSpeed1']
df_to_plot.plot(figsize=(12,8))
plt.xlabel('Time', fontsize=14)
plt.ylabel('Amplitude',fontsize=14);
plt.legend(fontsize=10)
name = path_pictures + 'perturbed_vs_unperturbed.png'
plt.savefig(fname=name)
\end{minted}
\label{lst:illu_perturbed_data}
\caption{Code used to produce plot in figure \ref{fig:illu_perturbed_data}}
\end{listing}

\begin{listing}
\begin{minted}
[
frame=lines,
framesep=2mm,
baselinestretch=1.2,
fontsize=\footnotesize,
linenos
]
{Python}
cumulative_explained_variance = {}
for turb in list(dict_perturbed_norm_dfs.keys()):
    X = dict_perturbed_norm_dfs[turb].values
    pca = PCA().fit(X)  # project from 64 to 2 dimensions
    cumulative_explained_variance[turb] = np.cumsum(pca.explained_variance_ratio_)

#fig, ax = plt.subplots(7,2)
plt.figure(figsize=(12,8))
for turb in list(cumulative_explained_variance.keys()):
    plt.plot(cumulative_explained_variance[turb], label=turb)
    
# plt.title('Explained variance vs. Number of PCA components', fontsize=20)
plt.xlabel('Number of components', fontsize=14)
plt.ylabel('Cumulative explained variance',fontsize=14);
plt.legend(fontsize=10)
name = path_pictures + 'explained_variance_perturbed.png'
plt.savefig(fname=name)
\end{minted}
\label{lst:cum_exp_var_pert}
\caption{Code used to calculate cumulative explained variance of perturbed data, and produce plot in figure \ref{fig:cum_exp_var_pert}}
\end{listing}

\begin{listing}
\begin{minted}
[
frame=lines,
framesep=2mm,
baselinestretch=1.2,
fontsize=\footnotesize,
linenos
]
{Python}
mse_pert = {}
for turb in list(dict_perturbed_norm_dfs.keys()):
    Xstd = dict_perturbed_norm_dfs[turb].values
    pca = PCA(n_components=2)
    pca.fit(Xstd)
    P=pca.components_
    T = Xstd.dot(P.T)
    Xhat = np.dot(T[:,0].reshape(-1,1),P[0,:].reshape(-1,1).T)
    mse_pert[turb] = (np.square(Xstd - Xhat)).sum(axis=1)

plt.figure(figsize=(12,8))
for turb in list(mse_pert.keys()):
    plt.plot(mse_pert[turb], label=turb)
    
# plt.title('Reconstruction Error when using two principal components', fontsize=20)
plt.xlabel('Time', fontsize=14)
plt.ylabel('Mean square reconstruction error',fontsize=14);
plt.legend(fontsize=10)
name = path_pictures + 'reconstruction_error_perturbed.png'
plt.savefig(fname=name)
\end{minted}
\label{lst:recon_error_pert}
\caption{Code used to calculate reconstruction error using perturbed data, and produce plot in figure \ref{fig:recon_error}}
\end{listing}

\begin{listing}
\begin{minted}
[
frame=lines,
framesep=2mm,
baselinestretch=1.2,
fontsize=\footnotesize,
linenos
]
{Python}
plt.figure(figsize=(12,8))
plt.plot(mse['WTUR12'], label='Original reconstruction error')
plt.plot(mse_pert['WTUR12'], label='Perturbed reconstruction error')
    
# plt.title('Reconstruction Error when using two principal components', fontsize=20)
plt.xlabel('Time', fontsize=14)
plt.ylabel('Mean square reconstruction error',fontsize=14);
plt.legend(fontsize=10)
name = path_pictures + 'pert_vs_unpert_reconstruction_error.png'
plt.savefig(fname=name)
\end{minted}
\label{lst:recon_error_pert_vs_unpert}
\caption{Code used to produce plot in figure \ref{fig:recon_error_pert_vs_unpert}}
\end{listing}