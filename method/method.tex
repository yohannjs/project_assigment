\newpage
\chapter{Method}

\section{Search Terms}
To find the relevant literature on the subjects of interest the search engine Oria was used to search the university library of the NTNU. 
Oria was preferred over other search engines such as Google Scholar because Oria allows one to combine multiple search terms in unison using ''AND'' or ''OR'', and because it allows the user to choose whether the search term should be in the title, subject, or other parts of the articles. 
Table \ref{tab:search_results} summarizes the search results. 
The \textit{Title} and \textit{General content} columns show which terms were used in the different searches; which terms where required to be in the title, and which terms could be in the ''general content'', meaning any part of the article. 
Let ''$\times$'' represent the AND operator between two search terms, and ''$\wedge$'' represent the OR operator. 
The ''*'' operator means that the search will include any word beginning with the word before the star, e.g \textit{detect*} includes \textit{detection}, \textit{detecting}, \textit{detected}, etc. 
The $N_f$ and $N_i$ columns show how many results each search yielded, and how many articles from each search were included in the review, respectively. 
The method of choosing which articles to include is outlined in section \ref{sec:screening}. \bigskip 

\begin{table*}[h]
    \centering
    \ra{1.3}
    \begin{tabular}{ lllrr } 
        \toprule
        Nr. & Title terms & General terms & $N_r$ & $N_i$ \\
        \midrule
        1 & wind $\times$ turbine* $\times$ (monitor* $\wedge{}$ detect*) $\times$ review & None & 32 & 3 \\
        2 & wind $\times$ turbine* $\times$ (monitor* $\wedge{}$ detect*) & machine $\times$ learning & 100 & 47 \\ 
        3 & time $\times$ series $\times$ clustering & None & 219 & 46 \\ 
        %4 & (monitor* $\wedge{}$ detect*) $\times$ clustering & time $\times$ series & 187 & 21 \\
        \midrule
        \multicolumn{4}{r}{Total number of articles included} & 96 \\
        \bottomrule
    \end{tabular}
    \caption{Search results}
    \label{tab:search_results}
\end{table*}

Search 1 and 2 are used to find articles covering the first objective mentioned in section \ref{sec:objective}, and search 3 is ment to cover the second objective.
\section{Screening Method} \label{sec:screening}
To make sure that the articles used were relevant, the review is limited to articles published in peer-reviewd journals, after January 2014. 
There were three levels of screening, screening of the title, abstract, and full article. 
Title-screening was primarily for seeding out duplicate articles returned from the search-engine. 
The screening of the abstract and full-article were to identify the articles that were not relevant for the review and exclude them. 
It has been a challenge to include enough literature to meet the objectives of the review, but also not more literature the author alone could handle. 
So although the objectives is to get an overview of the different time series clustering methods, and an overview of the current machine learning methods used for monitoring wind turbines, the author does not claim to have made a complete exhaustive summary of all the possible methods. \bigskip

Search number 1 was used to find existing literature reviews on condition monitoring of wind turbines. 
Three good literature reviews on the subject where found, and one good review on machine learning methods used for condition monitoring of wind turbines was found in search 2. 
So, when screening the remaining articles from search 3 the focus was to find articles not included in the aforementioned reviews, to complement them as well as possible. \bigskip

When screening articles from search 3 articles meeting one (or more) of the criteria outlined in table \ref{tab:excl_crit} were excluded from the review.

\begin{table*}
    \centering
    \ra{1.5}
    \begin{tabular}{p{0.5\textwidth}p{0.5\textwidth}}
        \toprule
        Criteria                                                                                               & Reason \\
        \midrule
        Primary goal is time-series prediction/forecasting.                                                    & It is outside the scope of this assignment.\\
        Uses subsequence time-series clustering methods.                                                       & It is outside the scope of this assignment.\\
        Time-series clustering is used only as a minor preprocessing step.                                     & Not considered relevant enough to the objectives of the review.\\
        Data used is not time series data.                                                                     & Not considered relevant enough to the objectives of the review.\\
        Paper does not actually use clustering algorithms.                                                     & Not considered relevant enough to the objectives of the review.\\
        The data used consists of image time series.                                                           & Data to different from data to be used in master thesis.\\
        The time-series clustering methods explored in the work are \textbf{not} model-based or feature-based. & The raw-data-based approach has been somewhat covered in Espen Waaga's master thesis, hence it is emmitted in this review.\\
        The specific model-based approach using the tail dependence of time series.                            & Method not found relevant enough for the data that will be used in the master thesis, more relevant for financial time series.\\
        \bottomrule
    \end{tabular}
    \caption{Exclusion criteria for articles in search 3}
    \label{tab:excl_crit}
\end{table*}
