\newpage
\chapter{Method}

\section{Search terms}
To find the relevant literature on the subjects of interest the search engine Oria was used to search the university library of the NTNU. Oria was preferred over other search engines such as Google Scholar because Oria allows one to combine multiple search terms in unison using ''AND'' or ''OR'', and because it allows the user to choose whether the search term should be in the title, subject, or other parts of the articles. The review will only consider articles published in peer-reviewed journals. Table \ref{tab:search_results} summarizes the search results. The \textit{Title} and \textit{General content} columns show which terms were used in the different searches; which terms where required to be in the title, and which terms could be in the ''general content'', meaning any part of the article. Let ''$\times$'' represent the AND operator between two search terms, and ''$\wedge$'' represent the OR operator. The ''*'' operator means that the search will include any word beginning with the word before the star, e.g \textit{detect*} includes \textit{detection}, \textit{detecting}, \textit{detected}, etc. The $N_f$ and $N_i$ columns show how many results each search yielded and how many articles from each search were included in the review, respectively. \bigskip 

\begin{table*}[h]
    \centering
    \ra{1.3}
    \begin{tabular}{ lllrr } 
        \toprule
        Nr. & Title terms & General terms & $N_r$ & $N_i$ \\
        \midrule
        1 & time $\times$ series $\times$ clustering & None & 219 & 121 \\ 
        2 & (monitor* $\wedge{}$ detect*) $\times$ clustering & time $\times$ series & 187 & 21 \\
        3 & wind $\times$ turbine* $\times$ (monitor* $\wedge{}$ detect*) $\times$ review & None & 32 & 3 \\
        4 & wind $\times$ turbine* $\times$ (monitor* $\wedge{}$ detect*) & machine $\times$ learning & 100 & 47 \\ 
        \midrule
        \multicolumn{4}{r}{Total number of articles included} & 193 \\
        \bottomrule
    \end{tabular}
    \caption{Search results}
    \label{tab:search_results}
\end{table*}

\section{Screening method}
To make sure that the articles used were relevant, the review is limited to articles published 2014 or later. Some older articles are included through backward snowballing for their historical importance. There were three levels of screening, screening of the title, abstract, and full article. Title-screening was primarily for seeding out duplicate articles returned from the search-engine. The screening of the abstract and full-article were to identify the articles that were not relevant for the review and exclude them. \bigskip

It has been a challenge to include enough literature to get a good overview of the different methods within time series clustering, but also not more literature the author alone could handle in the time available. So although the objectives is to get an overview of the different time series clustering methods, and an overview of the current machine learning methods used for monitoring wind turbines, the author does not claim to have made a complete exhaustive summary of all the possible methods. \bigskip

When screening articles from search one and two, articles meeting one (or more) of the following criteria were discarded: 
\begin{itemize}
    \item Primary goal is time-series forecasting. It is outside the scope of this assignment.
    \item Time-series clustering is used only as a minor preprocessing step. Not considered relevant enough to the objectives of the review.
    \item Data used is not time series data. Not considered relevant enough to the objectives of the review.
    \item Paper does not actually use clustering algorithms. Not considered relevant enough to the objectives of the review.
    \item The data used consists of image time series. Data to different from data to be used in master thesis.
    \item The time-series clustering methods explored in the work are \textbf{not} model-based or feature-based. The raw-data-based approach has been somewhat covered in Espen Waaga's master thesis, so work using this appraoch will be emmitted in this review.
\end{itemize}

Search number three was used to find existing literature reviews on condition monitoring of wind turbines. Three good literature reviews on the subject where found, and one good review on machine learning methods used for condition monitoring of wind turbines was found in search four. So, when screening the remaining articles from search number four the focus was to find articles not included in the aforementioned reviews, to complement them as well as possible. \bigskip
